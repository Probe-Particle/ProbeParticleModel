The quickest start is just to run an example script like test2.py from terminal with python.
\begin{shadedbox}
    \begin{lstlisting}[language=bash]
cd PyProbe_nonOrtho
python test2.py
    \end{lstlisting}
\end{shadedbox}


the script will automatically
\begin{enumerate}
\item   recompile the C++ code,
\item   load the library,
\item   load the input data example,
\item   setup and run the simulation
\item   and plot the results
\end{enumerate}

User input without scripting

Even though the code is supposed to be used as python library, and the main mean
of usage is to write python scripts, it can be also used fairly well without
editing any code, just by modification of input files.

There are three input files:
\begin{enumerate}

    \item params.ini - setup parametrs of simulation ( grid, sanning area, tip parameters, conversion and plotting )
    \item input.xyz - setup the geometry of the sample ( using e.g. standard xyz file format )
    \item atomtypes.ini - define radius and binding energy between pairs of atoms ( used for definition of Lenard-Jones model potential )

\end{enumerate}
