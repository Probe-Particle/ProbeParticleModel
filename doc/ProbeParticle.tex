\documentclass[english]{book}
\usepackage[utf8]{inputenc}
\usepackage[T2A,T1]{fontenc}
\usepackage[dvips]{graphicx} % Для начала работы с изображениями нужно подключить пакет graphicx, который обеспечивает их вставку в текст документа. Есть много драйверов для работы с изображениями, но мы будем использовать dvips - это позволит нам потом сравнительно легко и довольно просто конвертировать диплом из LaTeX в Word или OpenOffice через формат RTF, используя открытую программу latex2rtf. Для этого ваши рисунки нужно конвертировать в EPS, что делается программой convert из пакета imagemagick
\graphicspath{{images/}}
\usepackage{babel}
\usepackage{mathrsfs}
\linespread{1.3}
\usepackage[left=2cm, right=2cm,top=2cm,bottom=2cm]{geometry} % Разметка страницы

\usepackage{ gensymb }

\bibliographystyle{plain}

\usepackage[framemethod=tikz]{mdframed}
\usetikzlibrary{shadows}

\newmdenv[shadow=true,shadowcolor=black,font=\sffamily,rightmargin=8pt]{shadedbox}

\usepackage{listings}


\lstset{
basicstyle=\ttfamily
}

\usepackage[colorlinks,filecolor=blue,pagebackref]{hyperref} % Пакет hyperref рекомендуется загружать последним, поскольку он переопределяет многие команды LaTeX'а

\title{Probe Particle model}
\author{}
\begin{document}
\maketitle

\tableofcontents

\chapter{Introduction}
This is implementation of efficient and simple model for simulation of
High-resolution atomic force microscopy (AFM), scanning probe microscopy (STM)
and inelastic tunneling microscopy (IETS) images using classical forcefileds.

There are two versions of the code:

\begin{enumerate}
   \item currently developed Python/C++ version in PyProbe\_nonOrtho (branch master
    );
        to get quick esence of this model you can also try web interface hostet here: http://nanosurf.fzu.cz/ppr/
        for more details see wikipage:
        https://github.com/ProkopHapala/ProbeParticleModel/wiki
   \item Legacy fortran version in SHTM\_springTip2 (branch fortran );
        more detailed description o the fortran version is here:
        http://nanosurf.fzu.cz/wiki/doku.php?id=probe\_particle\_model
\end{enumerate}
\cite{phapalamechhighresol}
\cite{phapalaoriginhighresol}


\section{C++ \& Python version}

\subsection{Examples of results}


\subsection{How it works}

\begin{itemize}

    \item  Pauli-repulsion ( currently approximated by repulsive part of
    Lenard-Jones potential $r^{-12}$ )
    \item van der Waals attraction ( currently approximated by attractive part
    of Lenard-Jones potentia $r^{-6}$ )
    \item electrostatic ( currently can be computed as coulomb pairwise
    interaction between PP and point charges in centers of atoms of sample, or
    by reading electrostatic force-field obtained by derivative of sample
    Hartree potential as described in supplementary of this paper
    \cite{phapalaoriginhighresol}.

\end{itemize}

The computation of images is divided into two parts:
\begin{enumerate}
    \item Precompute vector Forcefield ( Fx(x,y,z), Fy(x,y,z), Fz(x,y,z) ) over
    sample and store it on a 3D-grid. ( see getLenardJonesFF and getCoulombFF
    functions for more details ). After individial components of forcefiled are
    sampled ( i.e. Lenard-Jones and electrostatic ) they are summed up to form
    one single total effective forcefield in which Probe-Particle moves.

    \item relax Probe-Particle attached to the tip under influence of the total
    effective forcefield. The relaxation of Probe-Particle is done using "Fast
    Inertial Realxation Engine" (FIRE) algorithm
    \cite{ebitzekstructrelaxmadesimple}. implemented in
    FIRE:move() function. In each step of relaxation the forcefield on the grid
    is interpolated in interpolate3DvecWrap function, considering periodic
    boundary condition. From python the relaxation is called as relaxTipStroke
    function providing 1D-array of tip positions, and obtaining back 1D-array of
    Probe-Particle position after the relaxation and of force between tip
    ProbePartcile and sample at that position. The lateral scan in (x,y) to
    obtained stack of images is done by calling relaxTipStroke at different
    (x,y) position of tip, where each call of relaxTipStroke does one approach
    along z-direction.

\end{enumerate}

\subsection{Why it is splitted like this ?}

This splitting of computation ( first sampling of forcefiled on the grid, and
than relaxation using interpolation) has several advantages over strightforward
computation of interaction on the fly during relaxation process.


\begin{itemize}

    \item \textbf{It is faster} - if there is ~100 atoms of sample, summing over
    each pairwise Lennard-Jones interactions, is much slower, than just
    interpolating the forcefield from the grid. Because force evaluation is done
    several-times for each voxel of the 3D scanning grid, it is more efficient
    to precompute it just once for each voxel, and then interpolate.

    \item \textbf{It is more flexible and general} - Decoupling of relaxation
    process and computation of forcefiled allows us to plug-in any forcefield.
    The original motivation was to used electrostatic forcefield obtained from
    Hartree potential from DFT calculation. However, it is not limited to that.
    We can plug in e.g. a derivative of potential energy of probe peraticle
    (i.e. Xe or CO ) obtained by scanning it in DFT code, in similar way as Guo
    did in \cite{chshguohighresolmodel} . The only limitation here is computational cost of obtaining
    such potential from ab-initio calculation.
\end{itemize}



\subsection{Code structure}
The code is divided into Python and C++, where performance intensive
computations are done in C++ and python is used as convenient scripting
interface for doing tasks like file I/O, plotting, memory management. For
binding of Python and C++ is used python ctypes library.


\subsubsection{Dependencies:}
\begin{itemize}
    \item \textbf{C++} : g++ ( tested with g++ (Ubuntu 4.8.1-2ubuntu1~12.04) 4.8.1 )
    \item \textbf{Python} : python ( 2.7.3 ), numpy (1.9.2), matplotlib ( 1.4.3 ), ctypes
    ( 1.1.0 )
\end{itemize}

\subsubsection{C++ source:}
\begin{itemize}
    \item  ProbeParticle.cpp - implementation of all performance intensive
    ProbeParticle model as dynamic library which can be called dynamically from
    python.
    \item Vec3.cpp,Mat3.cpp math subroutines for operations with 3D vectors and
    metrices
\end{itemize}

\subsubsection{Python source:}
\begin{itemize}
    \item ProbeParticle.py - Interface between C++ core and python ising
    C-types. Here are also defined some python rutines like:
    \begin{itemize}
        \item  conversion from Force to frequency shift ( Fz2df ),
        \item  evaluation of Lenard-Jones coeffitints ( getAtomsLJ )
        \item  copy sample geometry to simulate periodic boundary condition (
        PBCAtoms )
        \item automatic setup of imagining area acroding to geometry of
        nonperiodic sample ( autoGeom )
        \item default parameters of simulation ( params ) with subroutine to
        read this parameters from a file ( loadParams )
    \end{itemize}


    \item  test2.py, testServer2.py - two examples of python scripts with run
    actual computations using ProbeParticle library. The idea is that this files
    can be modified by user to match particular task
    \item basUtils.py - routines for loading of molecule geometry from
    xyz-format, ( loadAtoms ), finding bonds and other.
    \item Element.py and elements.py contains just parameters of atoms form
    periodic table ( like effective raidus, color for visualization etc. ). It
    is used by basUtils.py



\end{itemize}


\chapter{Quick Start}
The quickest start is just to run an example script like test2.py from terminal with python.
\begin{shadedbox}
    \begin{lstlisting}[language=bash]
cd PyProbe_nonOrtho
python test2.py
    \end{lstlisting}
\end{shadedbox}


the script will automatically
\begin{enumerate}
\item   recompile the C++ code,
\item   load the library,
\item   load the input data example,
\item   setup and run the simulation
\item   and plot the results
\end{enumerate}

User input without scripting

Even though the code is supposed to be used as python library, and the main mean
of usage is to write python scripts, it can be also used fairly well without
editing any code, just by modification of input files.

There are three input files:
\begin{enumerate}

    \item params.ini - setup parametrs of simulation ( grid, sanning area, tip parameters, conversion and plotting )
    \item input.xyz - setup the geometry of the sample ( using e.g. standard xyz file format )
    \item atomtypes.ini - define radius and binding energy between pairs of atoms ( used for definition of Lenard-Jones model potential )

\end{enumerate}


\chapter{Scripting}
\section{Examples}
Example of python scripts which setup simulation, run it, and plot results are
can be seen in test2.py and testServer2.py files.

Here we provide overview of various operations done in this example scripts with
a bit more detailed explanation than just inline-comments within the script
code. The topics and are discussed in order in which are done in the example the
script test2.py. The \textbf{order} of operations is sometimes important ( e.g. you
cannot allocate grid until you set it's size, you cannot do scan until you
sample forcefiled etc. )


\subsection{Import common libraries}

\begin{shadedbox}
    \begin{lstlisting}[language=python]
import os
import numpy as np
import matplotlib.pyplot as plt
import elements
import basUtils
    \end{lstlisting}
\end{shadedbox}




\subsection{Load ProbeParticle library}

C++ part of ProbeParticle library ( ProbeParticle.cpp ) is compiled into
ProbeParticle\_lib.so binary dynamic library. If we do any change in C++ code, we
have to recompile the library. This is done automatically during call of import
ProbeParticle if the file ProbeParticle\_lib.so is not pressent in the directory.
So, If we want force recompilation of the C++ dynamic library, we just delelete
the library. This can be done e.g. like this:


\begin{shadedbox}
    \begin{lstlisting}[language=python]
def makeclean( ):
    import os
    [ os.remove(f) for f in os.listdir(".") if f.endswith(".so") ]
    [ os.remove(f) for f in os.listdir(".") if f.endswith(".o") ]
    [ os.remove(f) for f in os.listdir(".") if f.endswith(".pyc") ]

makeclean( )  # force to recompile
import  ProbeParticle as PP

    \end{lstlisting}
\end{shadedbox}



\subsection{Setup system and simulation parameters}

One way to change simulation parameters is to change the default dictionary
PP.params. Which is by default e.g. defined like this:


\begin{shadedbox}
    \begin{lstlisting}[language=python]
params={
'PBC': False,
'gridN':       np.array( [ 150,     150,   50   ] ).astype(np.int32),
'gridA':       np.array( [ 12.798,  -7.3889,  0.00000 ] ),
'gridB':       np.array( [ 12.798,   7.3889,  0.00000 ] ),
'gridC':       np.array( [      0,        0,      5.0 ] ),
'moleculeShift':  np.array( [  0.0,      0.0,    -2.0 ] ),
'probeType':   8,
'charge':      0.00,
'r0Probe'  :  np.array( [ 0.00, 0.00, 4.00] ),
'stiffness':  np.array( [ 0.5,  0.5, 20.00] ),
'scanStep': np.array( [ 0.10, 0.10, 0.10] ),
'scanMin': np.array( [   0.0,     0.0,    5.0 ] ),
'scanMax': np.array( [  20.0,    20.0,    8.0 ] ),
'kCantilever'  :  1800.0,
'f0Cantilever' :  30300.0,
'Amplitude'    :  1.0,
'plotSliceFrom':  16,
'plotSliceTo'  :  22,
'plotSliceBy'  :  1,
'imageInterpolation': 'nearest',
'colorscale'   : 'gray',
}
   \end{lstlisting}
\end{shadedbox}

An alternative way is to load the parameters from a file using PP.loadParams.
This should be done before all other operations ( like definition and allocation
of sampling and scanning grid )




\begin{shadedbox}
    \begin{lstlisting}[language=python]
PP.loadParams( 'params.ini' ) # load parametes from ini file
   \end{lstlisting}
\end{shadedbox}


Then we should define system geometry ( positions of atoms and it types )
\begin{shadedbox}
    \begin{lstlisting}[language=python]
atoms    = basUtils.loadAtoms('input.xyz')
Rs       = np.array([atoms[1],atoms[2],atoms[3]]);
iZs      = np.array( atoms[0])
   \end{lstlisting}
\end{shadedbox}




ize and shape of sampling and scanning grid is normally set by parameters
PP.params['gridA'],'gridB','gridC','scanMin','scanMax'. However, in case of
non-periodic samples ( such as single molecule ) it is more conveninent let
program build proper sampling and scanning box around the molecule, instead of
defining the super-lattice verctors. This can be done by calling PP.autoGeom(
fitCell=True ) function like this:

Before the simulation the geometry of molecule should be shifted to proper
position with respect to sampling grid. The shift of the molecule can be set
either manually in PP.params['moleculeShift' ], or let program set this
parameter automatically using PP.autoGeom( shiftXY=True ).



\begin{shadedbox}
    \begin{lstlisting}[language=python]
if not PP.params['PBC' ]:
    PP.autoGeom( Rs, shiftXY=True,  fitCell=True,  border=3.0 )

Rs[0] += PP.params['moleculeShift' ][0]          # shift molecule so that we sample reasonable part of potential
Rs[1] += PP.params['moleculeShift' ][1]
Rs[2] += PP.params['moleculeShift' ][2]
Rs     = np.transpose( Rs, (1,0) ).copy()

   \end{lstlisting}
\end{shadedbox}

If we used periodic boundary condition, it is necessary to multiply geometry of
sample to neighboring unit cells. There is automatic procedure PP.PBCAtoms() to
do that:

\begin{shadedbox}
    \begin{lstlisting}[language=python]
if PP.params['PBC' ]:
    iZs,Rs,Qs = PP.PBCAtoms( iZs, Rs, Qs, avec=PP.params['gridA'],
    bvec=PP.params['gridB'] )
   \end{lstlisting}
\end{shadedbox}


Beside the geometry (position of atoms) the sample properties are defined also
by parameters of this atoms,such as Lenard-Jones parameters and charge. In
principle it is possible to set parameters (C6,C12 and Q ) of each atom
independently by hand (it is just array of numbers). However, more convenient
way is to read it from file of L-J parameters by PP.loadSpecies() for each atom
type and set the the parameters for each atom instance by PP.getAtomsLJ().
Charges for pairwise pointcharge Coulomb interaction are read from 5-th column
of geometry file.

\textbf{NOTE}: it is important to do this step after the multiplication of periodic
images to neighboring cells.

\begin{shadedbox}
    \begin{lstlisting}[language=python]
Qs       = np.array( atoms[4] )
FFparams = PP.loadSpecies        ( 'atomtypes.ini'  )
C6,C12   = PP.getAtomsLJ( PP.params['probeType'], iZs, FFparams )
   \end{lstlisting}
\end{shadedbox}

\subsection{Define and allocate arrays}
Do this before simulation, in case it will crash

\begin{shadedbox}
    \begin{lstlisting}[language=python]
dz    = PP.params['scanStep'][2]
zTips = np.arange( PP.params['scanMin'][2], PP.params['scanMax'][2]+0.00001, dz )[::-1];
ntips = len(zTips);
print " zTips : ",zTips
rTips = np.zeros((ntips,3))
rs    = np.zeros((ntips,3))
fs    = np.zeros((ntips,3))

rTips[:,0] = 1.0
rTips[:,1] = 1.0
rTips[:,2] = zTips

PP.setTip()

xTips  = np.arange( PP.params['scanMin'][0], PP.params['scanMax'][0]+0.00001, 0.1 )
yTips  = np.arange( PP.params['scanMin'][1], PP.params['scanMax'][1]+0.00001, 0.1 )
extent=( xTips[0], xTips[-1], yTips[0], yTips[-1] )
fzs    = np.zeros(( len(zTips), len(yTips ), len(xTips ) ));

nslice = 10;

FFparams = PP.loadSpecies        ( 'atomtypes.ini'  )
C6,C12   = PP.getAtomsLJ( PP.params['probeType'], iZs, FFparams )

print " # ============ define Grid "

cell =np.array([
PP.params['gridA'],
PP.params['gridB'],
PP.params['gridC'],
]).copy()

gridN = PP.params['gridN']

FF   = np.zeros( (gridN[2],gridN[1],gridN[0],3) )

   \end{lstlisting}
\end{shadedbox}




\subsection{Sample Lenard-Jones and electrostatic potential}

\begin{shadedbox}
    \begin{lstlisting}[language=python]

PP.setFF( FF, cell  )
PP.setFF_Pointer( FF )
PP.getLenardJonesFF( Rs, C6, C12 )

plt.figure(figsize=( 5*nslice,5 )); plt.title( ' FF LJ ' )
for i in range(nslice):
    plt.subplot( 1, nslice, i+1 )
    plt.imshow( FF[i,:,:,2], origin='upper', interpolation='nearest' )


withElectrostatics = ( abs( PP.params['charge'] )>0.001 )
if withElectrostatics:
    print " # =========== Sample Coulomb "
    FFel = np.zeros( np.shape( FF ) )
    CoulombConst = -14.3996448915;  # [ e^2 eV/A ]
    Qs *= CoulombConst
    #print Qs
    PP.setFF_Pointer( FFel )
    PP.getCoulombFF ( Rs, Qs )
    plt.figure(figsize=( 5*nslice,5 )); plt.title( ' FFel ' )
    for i in range(nslice):
        plt.subplot( 1, nslice, i+1 )
        plt.imshow( FFel[i,:,:,2], origin='upper', interpolation='nearest' )
    FF += FFel*PP.params['charge']
    PP.setFF_Pointer( FF )
    del FFel

plt.figure(figsize=( 5*nslice,5 )); plt.title( ' FF total ' )
for i in range(nslice):
    plt.subplot( 1, nslice, i+1 )
    plt.imshow( FF[i,:,:,2], origin='upper', interpolation='nearest' )
   \end{lstlisting}
\end{shadedbox}

\subsection{3D-Scan with ProbeParticle relaxation}
\begin{shadedbox}
    \begin{lstlisting}[language=python]
for ix,x in enumerate( xTips  ):
    print "relax ix:", ix
    rTips[:,0] = x
    for iy,y in enumerate( yTips  ):
        rTips[:,1] = y
        itrav = PP.relaxTipStroke( rTips, rs, fs ) / float( len(zTips) )
        fzs[:,iy,ix] = fs[:,2].copy()
    \end{lstlisting}
\end{shadedbox}



\subsection{Convert Fz -> df}
\begin{shadedbox}
    \begin{lstlisting}[language=python]
dfs = PP.Fz2df( fzs, dz = dz, k0 = PP.params['kCantilever'], f0=PP.params['f0Cantilever'], n=int(PP.params['Amplitude']/dz) )
    \end{lstlisting}
\end{shadedbox}



\subsection{Plot results of relaxed 3D-scan}
\begin{shadedbox}
    \begin{lstlisting}[language=python]
print " # ============  Plot Relaxed Scan 3D "

#slices = range( PP.params['plotSliceFrom'], PP.params['plotSliceTo'], PP.params['plotSliceBy'] )
#print "plotSliceFrom, plotSliceTo, plotSliceBy : ", PP.params['plotSliceFrom'], PP.params['plotSliceTo'], PP.params['plotSliceBy']
#print slices
#nslice = len( slices )

slices = range( 0, len(dfs) )

for ii,i in enumerate(slices):
    print " plotting ", i
    plt.figure( figsize=( 10,10 ) )
    plt.imshow( dfs[i], origin='upper', interpolation=PP.params['imageInterpolation'], cmap=PP.params['colorscale'], extent=extent )
    z = zTips[i] - PP.params['moleculeShift' ][2]
    plt.colorbar();
    plt.xlabel(r' Tip_x $\AA$')
    plt.ylabel(r' Tip_y $\AA$')
    plt.title( r"df Tip_z = %2.2f $\AA$" %z  )
    plt.savefig( 'df_%3i.png' %i, bbox_inches='tight' )

    \end{lstlisting}
\end{shadedbox}


\chapter{Technical details}
\section{Reference frame of the grids}

\subsection{Vertical alignment}

There are several non-intuitive hits about definition of coordinete system in
the code:


\begin{itemize}
    \item The code works in coordinate system where the origin of sampling grid
    is always in point (x,y,z)=(0,0,0). The geometry of sample should be alignet
    (i.e.) shifted to fit into this grid.

    \item In reality probe particle cannot approach to sample than ~3.5-4.0
    \AA ( because van der Waals radius of most atoms is typically 1.5-2.0
    \AA ). For this reason we do not want to sample region closer than
    1.0-2.0 \AA to the molecule where the Pauli repulsion would be
    extremely high, and where quantum mechanical interactions (chemical force)
    would be important.

    \item The region of sanning scanMax,scanMin is defined as a postion of tip
    apex (not probe particle). This means that probe particle can be outside the
    sampled forcefield grid. The real position of probe particle (i.e. the point
    where the force is interpolated from forcefield grid ) is shifted from this
    position by considerable distance ( ~ 4.0 \AA ). E.g. At the begining
    of each tip aproach the shift is r0Probe (relaxation is negligible).


\end{itemize}



Overview of relative position of various points in space is schematically
illustrate on following image:




\subsection{Lateral alignment}
For \textbf{ periodic systems} ( PBC=true ) the lateral position of sample is taken as it
is provided by user and geometry of supercell ( gridA, gridB, gridC) are defined
manually.

However, for \textbf{ non-periodic systems } it is convenient to use automatic routine
PP.autoGeom. This routine will make a bounding box around the sample ( according
to min an max coordinates of atoms ), and add some border (by default 3.0
Angstroem ) and than shift the molecule in the center of that box. The result
should be aligned like this:


\chapter{Meeting summary}
\section{13th of November 2015}

\begin{enumerate}
    \item \textbf{Cooding style}

    It is highly recommended to follow a generally accepted Python coding style.
    It would be usefull for an easy sharing of the code. More information can be found
    here: https://google.github.io/styleguide/pyguide.html

    \item \textbf{Documentation}

    It is required to put a docstring after each function definition. Points
    needed to be described are the following:
    \begin{enumerate}
        \item What function is doing

        \textit{Function checks ...}

        \item Input parameters description

        \textit{:param x: is x coordinate of the particle}

        \item Output parameters description:

        \textit{ :return: a boolean, true if...}
    \end{enumerate}

    After having created the docstring generation of the documentation can be done automatically

    \item \textbf{Standardized output}

    It must be usefull to create a function, which prints some ouput data in a
    standardized format.

    \item \textbf{Modular structure of the code}

    In order to have in future as less as possible problems related to the code
    design we need to define the structure as clear as possible. Because if later some other
    codes will use the  ProbeParticle, any changes of its strucure like
    output format etc. will end up in a big mess.

    Possible features that code should have:
        \begin{enumerate}
            \item Using of another codes to compute $F_z$ has to be
            straightforward

            \item Independent steps of calculations should be done by separate scripts.


        \end{enumerate}


    \item \textbf{Unexpected internal changes of the structure}

    Any changes of the molecule (its position, orientation, ...) must not be
    used as default operations, but optional things with a clear output of what
    have been done. This will prevent from possible confusions between what user
    expected to obtain and what he has obtained.


\end{enumerate}




\newpage
\bibliography{biblio}


%put your text here
\end{document}
