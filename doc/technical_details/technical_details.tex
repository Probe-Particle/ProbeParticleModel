\section{Reference frame of the grids}

\subsection{Vertical alignment}

There are several non-intuitive hits about definition of coordinete system in
the code:


\begin{itemize}
    \item The code works in coordinate system where the origin of sampling grid
    is always in point (x,y,z)=(0,0,0). The geometry of sample should be alignet
    (i.e.) shifted to fit into this grid.

    \item In reality probe particle cannot approach to sample than ~3.5-4.0
    \AA ( because van der Waals radius of most atoms is typically 1.5-2.0
    \AA ). For this reason we do not want to sample region closer than
    1.0-2.0 \AA to the molecule where the Pauli repulsion would be
    extremely high, and where quantum mechanical interactions (chemical force)
    would be important.

    \item The region of sanning scanMax,scanMin is defined as a postion of tip
    apex (not probe particle). This means that probe particle can be outside the
    sampled forcefield grid. The real position of probe particle (i.e. the point
    where the force is interpolated from forcefield grid ) is shifted from this
    position by considerable distance ( ~ 4.0 \AA ). E.g. At the begining
    of each tip aproach the shift is r0Probe (relaxation is negligible).


\end{itemize}



Overview of relative position of various points in space is schematically
illustrate on following image:




\subsection{Lateral alignment}
For \textbf{ periodic systems} ( PBC=true ) the lateral position of sample is taken as it
is provided by user and geometry of supercell ( gridA, gridB, gridC) are defined
manually.

However, for \textbf{ non-periodic systems } it is convenient to use automatic routine
PP.autoGeom. This routine will make a bounding box around the sample ( according
to min an max coordinates of atoms ), and add some border (by default 3.0
Angstroem ) and than shift the molecule in the center of that box. The result
should be aligned like this:
