This is implementation of efficient and simple model for simulation of
High-resolution atomic force microscopy (AFM), scanning probe microscopy (STM)
and inelastic tunneling microscopy (IETS) images using classical forcefileds.

There are two versions of the code:

\begin{enumerate}
   \item currently developed Python/C++ version in PyProbe\_nonOrtho (branch master
    );
        to get quick esence of this model you can also try web interface hostet here: http://nanosurf.fzu.cz/ppr/
        for more details see wikipage:
        https://github.com/ProkopHapala/ProbeParticleModel/wiki
   \item Legacy fortran version in SHTM\_springTip2 (branch fortran );
        more detailed description o the fortran version is here:
        http://nanosurf.fzu.cz/wiki/doku.php?id=probe\_particle\_model
\end{enumerate}
\cite{phapalamechhighresol}
\cite{phapalaoriginhighresol}


\section{C++ \& Python version}

\subsection{Examples of results}


\subsection{How it works}

\begin{itemize}

    \item  Pauli-repulsion ( currently approximated by repulsive part of
    Lenard-Jones potential $r^{-12}$ )
    \item van der Waals attraction ( currently approximated by attractive part
    of Lenard-Jones potentia $r^{-6}$ )
    \item electrostatic ( currently can be computed as coulomb pairwise
    interaction between PP and point charges in centers of atoms of sample, or
    by reading electrostatic force-field obtained by derivative of sample
    Hartree potential as described in supplementary of this paper
    \cite{phapalaoriginhighresol}.

\end{itemize}

The computation of images is divided into two parts:
\begin{enumerate}
    \item Precompute vector Forcefield ( Fx(x,y,z), Fy(x,y,z), Fz(x,y,z) ) over
    sample and store it on a 3D-grid. ( see getLenardJonesFF and getCoulombFF
    functions for more details ). After individial components of forcefiled are
    sampled ( i.e. Lenard-Jones and electrostatic ) they are summed up to form
    one single total effective forcefield in which Probe-Particle moves.

    \item relax Probe-Particle attached to the tip under influence of the total
    effective forcefield. The relaxation of Probe-Particle is done using "Fast
    Inertial Realxation Engine" (FIRE) algorithm
    \cite{ebitzekstructrelaxmadesimple}. implemented in
    FIRE:move() function. In each step of relaxation the forcefield on the grid
    is interpolated in interpolate3DvecWrap function, considering periodic
    boundary condition. From python the relaxation is called as relaxTipStroke
    function providing 1D-array of tip positions, and obtaining back 1D-array of
    Probe-Particle position after the relaxation and of force between tip
    ProbePartcile and sample at that position. The lateral scan in (x,y) to
    obtained stack of images is done by calling relaxTipStroke at different
    (x,y) position of tip, where each call of relaxTipStroke does one approach
    along z-direction.

\end{enumerate}

\subsection{Why it is splitted like this ?}

This splitting of computation ( first sampling of forcefiled on the grid, and
than relaxation using interpolation) has several advantages over strightforward
computation of interaction on the fly during relaxation process.


\begin{itemize}

    \item \textbf{It is faster} - if there is ~100 atoms of sample, summing over
    each pairwise Lennard-Jones interactions, is much slower, than just
    interpolating the forcefield from the grid. Because force evaluation is done
    several-times for each voxel of the 3D scanning grid, it is more efficient
    to precompute it just once for each voxel, and then interpolate.

    \item \textbf{It is more flexible and general} - Decoupling of relaxation
    process and computation of forcefiled allows us to plug-in any forcefield.
    The original motivation was to used electrostatic forcefield obtained from
    Hartree potential from DFT calculation. However, it is not limited to that.
    We can plug in e.g. a derivative of potential energy of probe peraticle
    (i.e. Xe or CO ) obtained by scanning it in DFT code, in similar way as Guo
    did in \cite{chshguohighresolmodel} . The only limitation here is computational cost of obtaining
    such potential from ab-initio calculation.
\end{itemize}



\subsection{Code structure}
The code is divided into Python and C++, where performance intensive
computations are done in C++ and python is used as convenient scripting
interface for doing tasks like file I/O, plotting, memory management. For
binding of Python and C++ is used python ctypes library.


\subsubsection{Dependencies:}
\begin{itemize}
    \item \textbf{C++} : g++ ( tested with g++ (Ubuntu 4.8.1-2ubuntu1~12.04) 4.8.1 )
    \item \textbf{Python} : python ( 2.7.3 ), numpy (1.9.2), matplotlib ( 1.4.3 ), ctypes
    ( 1.1.0 )
\end{itemize}

\subsubsection{C++ source:}
\begin{itemize}
    \item  ProbeParticle.cpp - implementation of all performance intensive
    ProbeParticle model as dynamic library which can be called dynamically from
    python.
    \item Vec3.cpp,Mat3.cpp math subroutines for operations with 3D vectors and
    metrices
\end{itemize}

\subsubsection{Python source:}
\begin{itemize}
    \item ProbeParticle.py - Interface between C++ core and python ising
    C-types. Here are also defined some python rutines like:
    \begin{itemize}
        \item  conversion from Force to frequency shift ( Fz2df ),
        \item  evaluation of Lenard-Jones coeffitints ( getAtomsLJ )
        \item  copy sample geometry to simulate periodic boundary condition (
        PBCAtoms )
        \item automatic setup of imagining area acroding to geometry of
        nonperiodic sample ( autoGeom )
        \item default parameters of simulation ( params ) with subroutine to
        read this parameters from a file ( loadParams )
    \end{itemize}


    \item  test2.py, testServer2.py - two examples of python scripts with run
    actual computations using ProbeParticle library. The idea is that this files
    can be modified by user to match particular task
    \item basUtils.py - routines for loading of molecule geometry from
    xyz-format, ( loadAtoms ), finding bonds and other.
    \item Element.py and elements.py contains just parameters of atoms form
    periodic table ( like effective raidus, color for visualization etc. ). It
    is used by basUtils.py



\end{itemize}
