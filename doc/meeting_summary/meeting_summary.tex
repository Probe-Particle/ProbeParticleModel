\section{13th of November 2015}

\begin{enumerate}
    \item \textbf{Cooding style}

    It is highly recommended to follow a generally accepted Python coding style.
    It would be usefull for an easy sharing of the code. More information can be found
    here: https://google.github.io/styleguide/pyguide.html

    \item \textbf{Documentation}

    It is required to put a docstring after each function definition. Points
    needed to be described are the following:
    \begin{enumerate}
        \item What function is doing

        \textit{Function checks ...}

        \item Input parameters description

        \textit{:param x: is x coordinate of the particle}

        \item Output parameters description:

        \textit{ :return: a boolean, true if...}
    \end{enumerate}

    After having created the docstring generation of the documentation can be done automatically

    \item \textbf{Standardized output}

    It must be usefull to create a function, which prints some ouput data in a
    standardized format.

    \item \textbf{Modular structure of the code}

    In order to have in future as less as possible problems related to the code
    design we need to define the structure as clear as possible. Because if later some other
    codes will use the  ProbeParticle, any changes of its strucure like
    output format etc. will end up in a big mess.

    Possible features that code should have:
        \begin{enumerate}
            \item Using of another codes to compute $F_z$ has to be
            straightforward

            \item Independent steps of calculations should be done by separate scripts.


        \end{enumerate}


    \item \textbf{Unexpected internal changes of the structure}

    Any changes of the molecule (its position, orientation, ...) must not be
    used as default operations, but optional things with a clear output of what
    have been done. This will prevent from possible confusions between what user
    expected to obtain and what he has obtained.


\end{enumerate}
